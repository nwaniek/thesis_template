% include the geometry package only if you really want to change geometry setup
% like perfectly centering the titlepage (though this is not required if the
% titlepage won't be the top page but is the first page inside the document
% (then you have to make sure to regard BCOR values)
\usepackage{geometry}

% with the url package, URLs can be entered like this: \url{...}, with the
% hyperref package they will become hyper links (and possibly colored)
\usepackage{url}

% to have equations always on the same indent, add the 'fleqn' option when
% loading the amsmath package
\PassOptionsToPackage{%
    centertags, %
    sumlimits, %
    intlimits, %
    namelimits}{amsmath}
\usepackage{amsmath}
\usepackage{amsthm}

% if you want to use the MinionPro font, comment the next line as it is
% incompatible with amssymb and amsfonts
\usepackage{amssymb, amsfonts}

% If you want to use lmodern package but have standardized math, include the
% fixmath package to get ISO31-0:1992 to ISO31-13:1992 compliant math output
\usepackage{fixmath}

\usepackage{titlesec}
\usepackage{lipsum}
\usepackage[pdftex]{graphicx}
\usepackage{tabularx}
\usepackage{longtable, lscape}


% the following package will inhibit the use of the TIKZ calc library
%\usepackage[all, warning]{onlyamsmath}

\usepackage{braket}
\usepackage{cancel}
\usepackage{empheq}
\usepackage{exscale}
\usepackage{icomma}
\usepackage{array}
\usepackage{colortbl}
\usepackage[square,numbers]{natbib}
\usepackage[titles]{tocloft}
\usepackage[pdftex, hyperfootnotes=false, pdfpagelabels, backref=page]{hyperref}
%\pdfcompresslevel=9
%\pdfadjustspacing=1
\usepackage{epstopdf}

\usepackage{float}    % floating figures, tables, etc.
\usepackage{rotating} % required for the sideways environment
\usepackage{dpfloat}
\usepackage{pbox}
%\usepackage{fltpage}

%
% Figure, Table, etc. captions and subfigures. caption package must come after
% floata and rotating
%
\usepackage[normal, labelfont=bf]{caption}
\usepackage[singlelinecheck=true]{subfig}

%
% Circumvent problems of marginpar with the marginnote package. Requires the
% document to be compiled more than just once to place the margins properly.
%
\usepackage{marginnote}


%
\usepackage{multirow}
\usepackage{ragged2e}
\usepackage{url}
%\usepackage{verbatim}
\usepackage[left]{eurosym}
\usepackage{epigraph}
\usepackage[dvipsnames]{xcolor}
\usepackage[automark, komastyle]{scrpage2}
\usepackage{scrtime}
\usepackage[bottom, stable, ragged, multiple]{footmisc}
\usepackage{microtype}
\usepackage{booktabs}
\usepackage{textcase}
\usepackage[boxed, lined, commentsnumbered]{algorithm2e}
\usepackage[toc, page]{appendix}
\usepackage{framed}
\usepackage{addlines}

%\usepackage{hyphenat}


% tikz/pgf plot and required packages
\usepackage{tikz}
\usepackage{pgfplots}
\usepackage{makecell}

% load additional tikz packages
\usetikzlibrary{decorations, arrows, matrix, trees, calc, shapes, positioning}

% pgf settings
%\pgfplotsset{compat=newest}
%\pgfplotsset{plot coordinates/math parser=false}


%
% consecutive footnote numbering
%
\usepackage{chngcntr}
\counterwithout{footnote}{chapter}

%\usepackage{remreset}

