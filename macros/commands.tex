\definecolor{shadecolor}{rgb}{1.0,0.7,0.5}

\newenvironment{todoenv}%
    {\begin{shaded}}%
    {\end{shaded}}

\newcommand{\todo}[1]{%
    \marginpar{%
            \vspace*{-1.5\baselineskip}%
            \begin{todoenv}%
            \textbf{TODO}:\\%
            #1%
            \end{todoenv}%

    }}%

\newcommand{\texttodo}[1]{%
    \begin{todoenv}%
        #1%
    \end{todoenv}%
    }%

\makeatletter
    \def\relativepath{\input@path}
\makeatother



% make things go over the whole page
\newlength{\wholemargin}\setlength{\wholemargin}{\marginparwidth}
\addtolength{\wholemargin}{\marginparpush}
\newlength{\wholewidth}\setlength{\wholewidth}{\textwidth}
\addtolength{\wholewidth}{\wholemargin}


\newlength{\wholefitwidth}\setlength{\wholefitwidth}{\marginparwidth}
\addtolength{\wholefitwidth}{\textwidth}

% Whole environment
\newenvironment{whole}{%
    \centering
    \strictpagechecktrue
    \begin{adjustwidth*}{0em}{-\wholemargin}
    \strictpagechecktrue
    \checkoddpage
    \ifoddpage%
        \raggedright
    \else%
        \raggedleft
    \fi%
}{%
    \end{adjustwidth*}
}


% figref + subfigref
\newcommand{\figref}[1]{\ref{fig:#1}}
\newcommand{\subfigref}[2]{\ref{fig:#1}\subref{fig:#1_#2}}


% import an image
\newcommand{\img}[1]{\subimport{img/}{#1}}

\newlength\figurewidth
\newlength\figureheight

% import tikz script
\newcommand{\tikzimp}[3]{{%
    \setlength\figurewidth{#2}
    \setlength\figureheight{#3}
    \subimport{tikz/}{#1.tikz}}}

% place a simple image to the marginpar
\newcommand{\mparimg}[1]{%
    \marginpar{%
    \def\svgwidth{\marginparwidth}
    \img{#1}
    }}

% place a figure with caption to the marginpar
\newcommand{\mparfig}[2]{%
    \marginpar{%
    \def\svgwidth{\marginparwidth}
    \img{#1}
    \captionof{figure}{#2}
    }}


% The same, but for TIKZ figures
\newcommand{\mpartikz}[1]{%
    \marginpar{
        \centering
        \subimport{tikz/}{#1}
    }}

\newcommand{\mpartikzfig}[2]{%
    \marginpar{
        \centering
        \subimport{tikz/}{#1}
        \captionof{figure}{#2}
    }}


\newcommand{\degrees}{{}^{\circ}}
\newcommand{\comment}[1]{}

% lazy bitch is lazy
\newcommand{\C}{\texttt{C}}
\newcommand{\ansic}{\texttt{ANSI C99}}
\newcommand{\Cpp}{\texttt{C++}}
\newcommand{\exr}{\texttt{OpenEXR}}
\newcommand{\openexr}{\texttt{OpenEXR}}
\newcommand{\exrtools}{\texttt{exrtools}}
\newcommand{\matexr}{\texttt{matexr}}
\newcommand{\exrflow}{\texttt{exrflow}}
\newcommand{\camlocpos}{\texttt{camlocpos}}
\newcommand{\censure}{\texttt{censure}}
\newcommand{\libcensure}{\texttt{libcensure}}
\newcommand{\blender}{\texttt{blender}}

\newcommand{\Ups}{\Upsilon}

\newcommand{\fundmat}{\mathbf{F}}
\newcommand{\essmat}{\mathbf{E}}
\newcommand{\calmat}{\mathbf{K}}


\newcommand{\expm}[1]{{}^{M}\exp{\left(#1\right)}}
\newcommand{\logm}[1]{{}^{M}\log{\left(#1\right)}}


% command to rotate the text sideways. used, for example, in tables. verttext = vertical text
%\newcommand{\verttext}[1]{\begin{sideways}\parbox{15mm}{#1}\end{sideways}}
\newcommand{\verttext}[1]{\begin{sideways}#1\end{sideways}}

\newcommand{\unit}[1]{\ensuremath{\, \mathrm{#1}}}

% specialized commands 
\newcommand{\mparoverview}[1]{
    \marginnote{
        \centering
        \subimport{tikz/mparoverview/}{#1.pgf}
    }}


% punctuation mark in equations. raise them vertically by increasing the
% raisebox value. usage: \eqpt{.} or \eqpt{,}, for example
\newcommand{\eqpt}[1]{\quad\raisebox{0pt}{#1}}


% tikz stuff
\newcommand{\polygon}[2]{%
  let \n{len} = {2*#2*tan(360/(2*#1))} in
 ++(0,-#2) ++(\n{len}/2,0) \foreach \x in {1,...,#1} { -- ++(\x*360/#1:\n{len})}}

